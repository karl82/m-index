\chapter{M-Index implementace v~jazyce Java}


\section{Výhody jazyka Java}

Jazyk Java je silně typovaný jazyk\@. Tento jazyk není interpretovaný,
výsledkem překladu je \emph{bytecode}\@. Tento \emph{bytecode} běží
v~tzv. Java Virtual Machine (JVM\nomenclature{JVM}{Java Virtual Machine})\@.
Díky JVM může kód psaný \emph{,,write once, run anywhere''}\@.

Hlavní klíčové vlastnosti, jak jazyka Java, JVM a JDK\nomenclature{JDK}{Java Development Kit}
jsou tyto: automatická alokace a dealokace paměti, objektovost, generika,
kompilace \emph{bytecode} do nativních instrukcí procesoru, bohatá
knihovna tříd v~základním JDK\ldots{}


\subsection{Automatická alokace a dealokace paměti}

V jazyce Java neexistují ukazatele, pouze reference. V~jazycích C/C++
se musí vývojář starat o~správu paměti pomocí funkcí \emph{malloc},
\emph{free} (jazyk C a C++), resp. operátorů \emph{new} a \emph{delete}
(jazyk C++)\@. V~jazyce Java tuto starost přebírá tzv. \emph{garbage
collector} (GC\nomenclature{GC}{garbage collector}), který sleduje
reference na jednotlivé alokovaní objekty v~paměti JVM. Pokud GC
zjistí, že na objekt nejsou žádné reference, uvolní pamět daného objektu%
\footnote{Podobná technika/pattern je součástí C++11 pomocí \emph{shared\_ptr},
kde se používá \emph{reference counting}%
}\@. Díky automatické správě paměti je vývoj výrazně rychlejší a výsledný
kód bezpečnější (nehrozí buffer overflow nebo underflow)


\subsection{Generika}

Obdobně jako v~C++ jsou \emph{templates}, jazyk Java má \emph{generics}.
Jsou zde rozdíly


\subsection{Kompilace \emph{bytecode} do nativních instrukcí procesoru}

JVM je v~heap procesor. Kvůli emulaci a virtualizaci procesoru je
běh samotného \emph{bytecode} pomalý. JVM proto umožňuje překompilovat
za běhu \emph{bytecode} do nativních instrukcí procesoru, nad kterým
JVM běží\@. Samozřejmostí je optimalizace při několika průchodech
kódem\@. Toto je velmi důležitá vlastnost, která při špatné interpretaci
zásadním způsobem ovlivní případná měření výkonnosti\@. Proto je
nutné několikrát spustit test během jednoho běhu JVM (více viz.~\prettyref{sub:M=00011B=000159en=0000ED-v=0000FDkonu-vJava})\@.


\subsection{Bohatá knihovna tříd v~základním JDK}

Java je dodávána s~rozsáhlou knihovnou JDK\@. Obsahuje vše nutné
pro práci se síťovou komunikací, práci s~textem -- regulární výrazy,
XML, RPC\nomenclature{RPC}{Remote Procedure Call}, soubory a souborový
systém a také velmi propracovanou knihovnou kontejnerů -- \emph{Java
Collections Framework }(JFC\nomenclature{JFC}{Java Collections Framework})%
\footnote{V~jazyce Java se používá pro termín\emph{ container} (C++) termín
\emph{collection} %
}\@. Její rozhraní (\emph{interfaces}) a jejich implementace jsou
použity při implementaci B-tree, \emph{cluster tree}.


\section{Měření výkonu v~Java\label{sub:M=00011B=000159en=0000ED-v=0000FDkonu-vJava}}


\section{Návrh a UML}


\section{Výsledky testů}


\section{Srovnání výkonu implementací v~C++, C\# a Java}
